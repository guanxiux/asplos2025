This chapter presents the detailed design for CacheInf to fulfill the functionality of tracking and reusing cached computation results and scheduling for actions among local computation, offloading or hybrid, with or without cache, to optimally reduce visual model inference latency for mobile robots.

\subsection{Identifying Reusable Computation Results}
To find and match similar local geometries between consecutive images in a stream of images $\textit{\textbf{I}}=\{I_1, I_2, ..., I_n\}$ to identify reusable cache, we use the standard image stitching procedure: given a pair of consecutive images $I_j$ and $I_{j+1}$, their key points and key point descriptors (or feature vectors) are computed and matched within a distance threshold of the feature vectors; then a homography matrix $M$ is computed based on the corresponding relationship between the key points on each image which minimizes the error.
The resulting homography matrix is then used to apply perspective transformation to each pixel in $I_j$ to form a new image $\hat{I}_{j+1}$ closest to $I_{j+1}$ as shown in Equation~\ref{eq: pt}, where $(u_j,v_j)$ and $(\hat{u}_{j+1},\hat{v}_{j+1})$ and pixel indices on $I_j$ and $\hat{I}_{j+1}$.
It is also depicted in the Feature Based Transformed Image in Fig.~\ref{fig:overview}.
Since the computation of local operators relies on local geometries, the same transformation can be applied to intermediate computation results of the following local operators.
\begin{equation}
    (\hat{u}_{j+1},\hat{v}_{j+1},1) = M \times (u_j,v_j,1)
    \label{eq: pt}
\end{equation}

While the above process minimizes error between $\hat{I}_{j+1}$ and $I_{j+1}$, the remaining difference between them are the areas of new information which are uncached and needed to be recomputed.
We filter and identify these areas by applying average pooling over the difference between $\hat{I}_{j+1}$ and $I_{j+1}$ and the pixels with computed difference greater than a preset threshold $N$ will be marked as needed to be recomputed as in Equation~\ref{eq:avg_pool}, where $u,v$ are the pixel indices.

\begin{equation}
    \textit{\textbf{uv}} = \{(u,v)|AveragePooling(|\hat{I}_{j+1} - I_{j+1}|)^{u,v} \geq N\}
    \label{eq:avg_pool}
\end{equation}

Suppose there are $Q$ pixels in \textit{\textbf{uv}} and $H$x$W$ total pixels in each image,
we define the cache ratio between $I_j$ and $I_{j+1}$ as 
\begin{equation}
    r_j = \frac{Q}{H\cdot W}
\end{equation}

\subsection{Sparse Local Operators}
From the above discussion, we have identified the pixels needed to recompute \textit{\textbf{uv}} and we suppose their corresponding features $f_{inp}$ are of size $B$x$C_1$x$Q$, along with the cached input defined as $I'_{inp}$ of size $B$x$C_1$x$H$x$W$.
Now we focus on how to compute the correct results based \textit{\textbf{uv}}, $f_{inp}$ and $I'_{inp}$.
There are mainly two kinds of local operators: element-wise local operators such as addition, subtraction, multiplication and division, which solely depends on the value of each element; and convolution local operators such as convolution, average pooling and max pooling, which is influenced by the surrounding areas (e.g., a 2D kernel) of each element .
We mainly focus on the latter type of local operators since the element-wise local operators can be viewed as a special case of convolution local operators where the surrounding area is of size one.

We first consider the scenario with dense input.
Assume an image (or feature map) $I_{inp}$ of size $B$x$C_1$x$H$x$W$, a convolution local operator $K$ with its kernel sized $C_2$x$C_1$x$K_1$x$K_2$, stride 1 and no padding and its output feature map $I_{out}$ of size $B$x$C_2$x$H'$x$W'$, then each of the value of the output feature map is determined by
\begin{equation}
    I_{out}^{i,j,k,l} = \sum_{c=1}^{C_1} \sum_{m=1}^{K_1} \sum_{n=1}^{K_2} K^{j,c,m,n} * I_{inp}^{i,c,k+m-1,l+n-1}, 
    \label{eq:kernel}
\end{equation}
Omitting the batch dimension and the channel dimension (first two dimension) of $I_{out}$, we can learn from Equation~\ref{eq:kernel} that an output value is determined by an area of $K_1$x$K_2$ on $I_{inp}$ and we define pixels in this area as
\begin{equation}
    P_{k,l} = \{(u,v)|k\leq u < k+K_1 \land l\leq v < l+K_2\}
    \label{eq:set}
\end{equation}
where $(k,l)$ is the pixels indices on $I_{out}$.

Moving to the sparse scenario, 
the indices of pixels on $I_{out}$ that have updated value with \textit{\textbf{uv}} as input would be 
\begin{equation}
\textit{\textbf{uv}}' = \{(k,l)|\exists P_{k,l}, s.t. P_{k,l}\cap \textit{\textbf{uv}} \neq \emptyset\}
\end{equation} 
which can be view as wrapping around pixels in \textit{\textbf{uv}} by $K_1$x$K_2$ and may involve pixels in $I'_{inp}$.

Note that $\textit{\textbf{uv}}$ and cached input $I'_{inp}$ are possibly in different planes determined by the homography matrix $M$.
We may transform the cached intermediates every time before computation, but it will unfortunately involve computation of the whole feature map and invalidate the acceleration of sparse computation.
Instead, during computation we query the original cached intermediates by transforming the pixel indices with $M$:
\begin{equation}
    F(i,j,u,v, I'_{inp}, f_{inp}) = \left\{
        \begin{aligned}
            f_{inp}^{i,j,u,v}  & , & (u,v) \in \textit{\textbf{uv}}, \\
            {I'}_{inp}^{i,j,G(u,v,M)} &, & (u,v) \notin \textit{\textbf{uv}}
        \end{aligned}
        \right. 
    \label{eq:query}
\end{equation}
where $G(u,v,M) = H^{-1}(M^{-1}\times H((u,v)))$ which transforms $(u,v)$ into the plane of cached input $I'_{inp}$, and $H(\cdot)$ and $H^{-1}(\cdot)$ means turning a vector to a homogeneous vectors and the opposite.
To minimize performance impact to update $I'_{inp}$, we update $I'_{inp}$ by transforming $I'_{inp}$ and merge it with $f_{inp}$ only after the whole computation process finishes, when the system is typically idle and waiting for the next input.

For $(u,v)\in\textit{\textbf{uv}}'$
\begin{equation}
    f_{out}^{i,j,u,v} = \sum_{c=1}^{C_1} \sum_{m=1}^{K_1} \sum_{n=1}^{K_2} K^{j,c,m,n}\cdot F(i,c,k+m-1,l+n-1, I'_{inp}, f_{inp})
\end{equation}
Until now we get the indices of the altered output values in output feature map $\textit{\textbf{uv}}'$ and the corresponding features $f_{out}$ which can then be passed to the subsequent computation.

Along the local operators where local geometries are preserved, we can repeat the above process by passing only the sparse features and their indices and do not need to merge the sparse features with cache.
When a non-local operator is met (e.g., matrix multiplication), we transform its cached input with $M$ and merge $f_{inp}$ into the transformed input according to their sparse indices $\textit{\textbf{uv}}$, which recovers the correct geometries of the whole feature map.

Also, to save memory consumption of cached intermediates, notice that the above process is basically wrapping the sparse pixels with the kernel size $K_1$x$K_2$ and computing on the wrapped pixels, we can merge the query process in Equation~\ref{eq:query} of multiple convolution local operators into the first convolution local operator.
For example, if a next operator is a convolution local operator with kernel size $K'_1$x$K'_2$, we can wrap the sparse pixels with an extended kernel size $(K_1+K'_1)$x$(K_2+K'_2)$ in the first local operator, and the wrapping process of the next operator is skipped (we refer to this process as merging cache).
In this case, the cache for the input of the next operator is needless and can be excluded to save memory consumption and the reduced number of cached input further leverages the cost to update $I'_{inp}$.


\subsection{Cache-Aware Scheduling}
In the above discussion we have analyzed the opportunity for visual model inference acceleration by reusing previous computation result.
In a edge-cloud collaborative system as CacheInf, reusing previous computation result has the potential to not only reduce transmission time by reducing transmission data volume, but also reduce local computation time by shrinking computation size using sparse local operators, and the extend of such reduction is determined by cache ratio of the current input $r$.
Similar to Hybrid-Parallel~\cite{sun2024hybridparallel}, we define all the operators involved in a visual model as $\textit{\textbf{O}} = \{o_1, o_2,...,o_n\}$ and the portion of locally executed input of each operator will be $\textit{\textbf{X}}=\{x_1, x_2, ..., x_n\}$, $0\le x_i \ge 1$ and $1-x_i$ represents the the portion of input executed on the GPU server.
The indices of local operators is defined as $\textit{\textbf{O}}_l$.

In CacheInf, while offloading, we transmit the sparse features together with their indices encoded into a bit-mask and the transmission data volume is almost inversely proportional to cache ratio: assume wholely offloading computation of a visual model from a layer will require transmission of $m=B$x$C$x$H$x$W$ element and each element is a float32 number, summing up to 4m bytes data; in the cached case with cache ratio $r$, the data volume needed to transmit will be $(1-r)4m + \frac{m}{8BxC}$ bytes, where the latter term is the bit-mask of $H$x$W$ encoding the indices of the transmitted pixels on the whole feature map.
When integrating Hybrid-Parallel that would slice the sparse features, we simply slice the sparse features and only encode the bit-mask for indices of the slice for the server, while preserving the features and indices for the robot.

However, the local computation time acceleration with sparse local operators has a complex relationship with cache ratio, which is affected by the operator implementation, gpu structure, input shapes and so on.
Thus we profile such relationship by altering the cache ratio and $x_i$ and recording the average execution time for every operator involved in the visual model and we define the profile result as a function $T_c(o_i, x_i, r)$ for the robot ($c$ means client) and $T_s(o_i, x_i, r)$ for the server, which returns the execution time of operator $o_i$ under $x_i$ with cache ratio $r$.
We also profile the time cost to update cached input for each local operator and get $U_r(o_i, x_i, r)$ and $U_s(o_i, x_i, r)$.
Note that for non-local operators (indices $\textit{\textbf{O}}_{nonlocal}=\{i|1\le i\le n \land i \notin \textit{\textbf{O}}_l\}$), we constraint that both $T(\cdot)$ and $U(\cdot)$ returns time of $x_i$ to be either 0 or 1 and $r$ to be 0.

\subsubsection{Schedule to Merge Cache}
With the above setup, the first problem to solve will be the choices of merging the cache of sparse local operators to further accelerate computation while saving memory consumption.
We define the indices of the chosen operators to cache their input as $\hat{\textit{\textbf{O}}}_c$ and the resulting reduction of cache ratio (since extra input will be included) of each operator as $o_I$ to be $R(\hat{\textit{\textbf{O}}}_c, o_i)$.
Since these choices will determine the operators that will cache their input and will be reused across different inference, these choices should be fixed during the whole inference task.
Thus we start by considering only the worst case where offloading is not possible and $\forall x_i\in \textit{\textbf{X}}, x_i=1$.
In this case the execution time of every operator will be $T_c(o_i, 1, r-R(\hat{\textit{\textbf{O}}}_c, o_i))$, and the optimization problem will be 
\begin{equation}
    \min\limits_{\hat{\textit{\textbf{O}}}_c \subset \textit{\textbf{O}}_l} \frac{1}{w} \sum_{i=1}^n \sum_{j=0}^w T_c(o_i, 1, r_j+R(\hat{\textit{\textbf{O}}}_c, o_i)) + U_{sum}(\hat{\textit{\textbf{O}}}_c, r_j)
    \label{eq:cached op}
\end{equation}
where $r_j = \frac{j}{w}$ with $w>1$ is the possible cache ratio considered (we empirically set $w$ to 10), and $U_{sum}(\hat{\textit{\textbf{O}}}_c, r) = \sum\limits_{k \in \hat{\textit{\textbf{O}}}_c} U(o_k, x_k, r)$ is the total time to update cache of operators in $\hat{\textit{\textbf{O}}}_c$.

Solving of this optimization problem seeks the optimal choices of cache operators $\hat{\textit{\textbf{O}}}_c$ that minimizes local execution time averaged across all possible cache ratio.
Note that we do not need to explicitly consider memory consumption because the latter term in Equation~\ref{eq:cached op} will naturally reduce the number of cached operators and favor operators with smaller size of input and thus shorter time to update cache.

\subsubsection{Schedule of Offloading}
Finally, we are combining all the above components to schedule for computation and offloading in a cache-aware way to optimize end-to-end inference latency for robotic visual models.
With Hybrid-Parallel integrated, cache can exists partially both at the robot and the server and we analyze the cache ratio on robot $r_c$ and the cache ratio $r_s$ on server by enquiry the current cached pixels with the previous slice of input (i.e., $x_i$).
For an $x_i$, we define the minimum portion of locally executed input of its parent operators (i.e., operators whose output is the input of $o_i$) as $x'_i$ and different between $x_i$ indicates offloading to/from the server.
For every operator $o_i \in \textit{\textbf{O}}$ involved in a visual model, we define its finishing time since the first operator starts executing as $t_i^{c}$ on the robot ($c$ means client) and $t_i^{s}$ on server.
Note that we insert an operator to analyze the image input as the first operator which will always be executed at the robot.

We can have the finish time of each operator on the robot and the server as the following, where $D(o_i, x'_i-x_i,r)$ is the data volume needed to be transmitted at operator $o_i$ with cache ratios $r_c$ and $r_s$ and $b$ is the estimated bandwidth:
\begin{equation}
    t_i^{c} = \left\{
        \begin{aligned}
            &T_c(o_i, x_i, r_c-R(\hat{\textit{\textbf{O}}}_c, o_i)),&&i=1\\
            &t_{i-1}^{c} + T_c(o_i, x_i, r_c-R(\hat{\textit{\textbf{O}}}_c, o_i)),&&1<i\le n \land x_i\le x'_i \\
            &max(T_c(o_i, x_i, r_c-R(\hat{\textit{\textbf{O}}}_c, o_i)) +\\
            & \enspace \enspace t_{i-1}^{c}, \enspace \frac{1}{b}D(o_i, x'_i-x_i,r)+ t_i^{s}),&&1<i\le n \land x_i >x'_i
        \end{aligned}
    \right.
    \label{eq:robot}
\end{equation}

\begin{equation}
    t_i^{s} = \left\{
        \begin{aligned}
            &0,&&i=1\\
            &t_{i-1}^{s} + T_s(o_i, 1-x_i, r_s-R(\hat{\textit{\textbf{O}}}_c, o_i)),&&1<i\le n \land x_i\ge x'_i \\
            &max(T_s(o_i, 1-x_i, r_s-R(\hat{\textit{\textbf{O}}}_c, o_i)) + \\
            & \enspace \enspace t_{i-1}^{s}, \enspace \frac{1}{b} D(o_i, x'_i-x_i,r_s)+ t_i^{c}),&&1<i\le n \land x_i<x'_i
        \end{aligned}
    \right.
    \label{eq:server}
\end{equation}
The first two rows of both Equation~\ref{eq:robot} and~\ref{eq:server} describes the scenarios where either the robot or the server does not need to receive data from the opposite side and thus the finishing time of this operator only depends on its local execution time.
The third row instead describes the opposite scenarios, where either the robot or the server needs to receive data from the opposite side (e.g., $x_i > x'_i$ for the robot) and have to wait until the same operator to finish computing at the opposite side and then be transmitted at bandwidth $b$.

With the above statements, optimizing the end-to-end inference latency for the visual model with cache enabled at a given bandwidth $b$ and cache ratios $r_c$ and $r_s$ is to solve
\begin{equation}
    \begin{aligned}
        \min\limits_{\forall 1\le i \le n, 0\le x_i\le 1} & t_n^{c} \\
         s.t. \enspace \enspace&  x_1 = x_n = 1 \\
         & \forall j \in \textit{\textbf{O}}_{nonlocal}, x_j (mod \enspace 1) = 0 \\
         & \forall j \notin \hat{\textit{\textbf{O}}}_c, x_j = x'_j
    \end{aligned}
    \label{eq:e2e}
\end{equation}
In Equation~\ref{eq:e2e}, the first two constraints ensure that inference output will finally be located at the robot and non-local operators will always have full input; the third constraint ensures that offloading will not happen within an operator whose cached is merged into the cache of other operators, since we cannot recover the operator's whole feature map.
We solve both optimization problems in Equation~\ref{eq:cached op} and~\ref{eq:e2e} with the differential evolution algorithm~\cite{qin2008differential} and store the solutions of different bandwidth and cache ratios of Equation~\ref{eq:e2e} in a dictionary referred to as $Schedule$.

The resulting algorithms of CacheInf at both the robot and the server are presented in Algorithm~\ref{alg:client} and Algorithm~\ref{alg:server}

\begin{algorithm}[htbp]
    \caption{\small CacheInfClient\label{alg:client}}
    \SetKwInput{KwInput}{Input}                % Set the Input
    \SetKwInput{KwOutput}{Output}              % set the Output
    \DontPrintSemicolon
      \KwInput{\small A continual sequence of video images $\textit{\textbf{I}}$; DNN model $M$}
      \KwOutput{\small The inference results $ret$ on each image in $\textit{\textbf{I}}$ }
    %   \KwData{\small input of $i_{th}$ layer $Z_{i}$; schedule plan of $i_{th}$ layer under the $b$ bandwidth $X_{i}^{b}$, $M_{i}^{b}$,$N_{i}^{b}$}
       \tcp{\footnotesize profile}
    
        $T_c, U_c = Profile(M)$
    
        $Send(M, T_c, U_c)$
    
        $Schedule, \hat{\textit{\textbf{O}}}_c = Receive()$

        $Cache = InitCache(\hat{\textit{\textbf{O}}}_c)$

       \tcp{\footnotesize inference}

       \ForEach{$I \enspace in \enspace \textit{\textbf{I}}$ \enspace}{
            $b = EstimateBandwidth()$

            $r_c, r_s = AnalyzeCacheRatio(I, Cache[1])$

            $x, x' = Schedule[b,r_c,r_s]$

            $inp, M = IdentifyCache(I, Cache[1])$

            $ Send(b, r_c, r_s, M) $

            \ForEach{$i = 1, 2, ..., n $ \enspace}{
                \If{$i \in \textit{\textbf{O}}_{nonlocal}$ and $x_i>0$}{
                    $inp = DenseRecover(inp, Cache[i],M)$
                }
                \If{$x_i < x'_i$}{
                    $inp, inp' = Slice(inp, x_i, x'_i)$

                    $Send(inp')$
                }
                \ElseIf{$x_i > x'_i$}{
                    $inp = Merge(inp, Receive())$
                }

                \If{$x_i>0$}{

                    \If{$i\in \hat{\textit{\textbf{O}}}_c$}{
                        $inp = SparseGather(inp, Cache[i], M)$

                        $UpdateCache(Cache[i], inp, M)$
                    }
                    $inp = Execute(o_i, inp)$ 
                }

            }
            $ret[I] = inp$
       }
      
\KwRet{$ret$}
\end{algorithm}


\begin{algorithm}[htbp]
    \caption{\small CacheInfServer\label{alg:server}}
    % \SetKwInput{KwInput}{Input}                % Set the Input
    % \SetKwInput{KwOutput}{Output}              % set the Output
    % \DontPrintSemicolon
    %   \KwInput{\small A continual sequence of video images $\textit{\textbf{I}}$; DNN model $M$}
    %   \KwOutput{\small The inference results $ret$ on each image in $\textit{\textbf{I}}$ }
    % %   \KwData{\small input of $i_{th}$ layer $Z_{i}$; schedule plan of $i_{th}$ layer under the $b$ bandwidth $X_{i}^{b}$, $M_{i}^{b}$,$N_{i}^{b}$}
       \tcp{\footnotesize profile and compute schedule at the server}
    
        $ M,T_c, U_c = Receive()$

        $T_s, U_s = Profile(M)$

        $Schedule, \hat{\textit{\textbf{O}}}_c = ComputeSchedule(T_s, U_s, T_c, U_c)$

        $ Send(Schedule, \hat{\textit{\textbf{O}}}_c) $

        $Cache = InitCache(\hat{\textit{\textbf{O}}}_c)$

       \tcp{\footnotesize inference}

        \While{True}{
            $ b, r_c, r_s, M = Receive()$ 

            $x, x' = Schedule[b,r_c,r_s]$

            \ForEach{$i = 1, 2, ..., n $ \enspace}{
                \If{$i \in \textit{\textbf{O}}_{nonlocal}$ and $x_i<1$}{
                    $inp = DenseRecover(inp, Cache[i],M)$
                }
                \If{$x_i > x'_i$}{
                    $inp, inp' = Slice(inp, x_i, x'_i)$

                    $Send(inp')$
                }
                \ElseIf{$x_i < x'_i$}{
                    $inp = Merge(inp, Receive())$
                }

                \If{$x_i < 1$}{

                    \If{$i\in \hat{\textit{\textbf{O}}}_c$}{
                        $inp = SparseGather(inp, Cache[i], M)$

                        $UpdateCache(Cache[i], inp, M)$
                    }
                    $inp = Execute(o_i, inp)$ 
                }

            }
       }
\end{algorithm}
    
    
% \begin{algorithm}[htbp]
% \caption{\small CacheInfServer\label{alg:server}}
% \SetKwInput{KwInput}{Input}                % Set the Input
% \SetKwInput{KwOutput}{Output}              % set the Output
% \DontPrintSemicolon
%     \KwData{\small input of $i_{th}$ layer $Z_{i}$; schedule plan of $i_{th}$ layer under the $b$ bandwidth $Y_{i}^{b}$, $M_{i}^{b}$,$N_{i}^{b}$}
%     \tcp{\footnotesize profile phase on server}    
%     $model,info\_robot = ReceiveFromClient()$    
%     $info\_server = ProfileModel(model)$    
%     $X, Y, M, N = LOSS(info\_robot,info\_server)$    
%     $SendToClient(X,M,N)$    
%     \tcp{\footnotesize inference phase on robot}    
%     $b = PredictsBandwidth()$    
%     $Z_{0} = \emptyset$    
%     \ForEach {$i_{th} \quad layer \quad in\quad model$ }{    
%     \If{$M_{i}^{b} \neq \emptyset$}
%     {    
%     $SendToClient(Z_{i}, M_{i}^{b})$        
%     }    
%     \If{$N_{i}^{b} \neq \emptyset$}
%     {      
%     $Z_{i} = combine(Z_{i}, ReceiveFromClient())$        
%     }    
%         \If{$Y_{i}^{b} \neq \emptyset$ and $Z_{i} \neq \emptyset$}
%     {    
%         $Z_{i+1} = compute(Z_{i},Y_{i}^{b})$        
%     }
%     \Else{    
%     $Z_{i+1} = \emptyset$
%     }    
%     }       
% \end{algorithm}





